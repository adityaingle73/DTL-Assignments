\documentclass[12pt]{article}
\author{\large Aditya Ingle \\ MIS:112103050}
\date{\today}
\begin{document}
\raggedright
\title{\huge \bf ASSIGNMENT 4}
\maketitle
\pagenumbering{gobble}
\newpage
\pagenumbering{roman}
\newpage
\section{Reasons for studying concepts of programming language}
\subsection{Increased capacity to express ideas}
\subparagraph{}It is natural for students to wonder how they will benefit from the study of programming language concepts~\cite{Paper}. After all, many other topics in computer science
are worthy of serious study~\cite{Latex}. The following is what we believe to be a compelling
list of potential benefits of studying concepts of programming languages.It is widely believed that the depth at which
people can think is influenced by the expressive power of the language in
which they communicate their thoughts ~\cite{aa,agarwal}. Those with only a weak under-
standing of natural language are limited in the complexity of their thoughts,
particularly in depth of abstraction. In other words, it is difficult for people
to conceptualize structures they cannot describe, verbally or in writing ~\cite{agarwal,Kshir,rucha}.


\begin{thebibliography} {}

\bibitem {aa}agarwal Ugale.,N. Bhatt.,Enginnering Physics SMA,978-1-6654-1703-7/21.

\bibitem{agarwal} S. Agarwal.,N. Bhatt.,Enginnering Physics SMA,978-1-6654-1703-7/21.

\bibitem{Latex} Latex,Latex ,IEEE

\bibitem{Kshir} S. Kshirsagar.,N. Bhatt.,Enginnering Physics SMA,978-1-6654-1703-7/21.

\bibitem{rucha} S. Agarwal.,S. Kshirsagar.,Enginnering Physics SMA,978-1-6654-1703-7/21.

\bibitem{Paper} ,Research Paper,2022,IEEE.
\end{thebibliography}
\newpage
\subsection{Better understanding of the significance of implementation}
\subparagraph{}In learning the concepts of programming languages, it is both interesting and necessary to touch on the implementation issues that affect those concepts [1] .In some cases, an understanding of implementation issues leads to an understanding of why languages are designed the way they are. In turn, this knowledge leads to the ability to use a language more intelligently, as it was designed to be used [2-4] .We can become better programmers by understanding the choices among programming language constructs and the consequences of those choices.Certain kinds of program bugs can be found and fixed only by a programmer who knows some related implementation details [2] . Another benefit of understanding implementation issues is that it allows us to visualize how a computer executes various language constructs. In some cases, some knowledge of implementation issues provides hints about the relative efficiency of alternative constructs that may be chosen for a program. For example, programmers who know little about the complexity of the implementation of subprogram calls often do not realize that a small subprogram that is frequently called can be a highly inefficient design choice.


\begin{thebibliography} {}

\bibitem{Boney96} Boney, L., Tewfik, A.H., and Hamdy, K.N., ``Digital
Watermarks for Audio Signals," \emph{Proceedings of the Third IEEE
International Conference on Multimedia}, pp. 473-480, June 1996.
\bibitem{MG} Goossens, M., Mittelbach, F., Samarin, \emph{A LaTeX
Companion}, Addison-Wesley, Reading, MA, 1994.
\bibitem{HK} Kopka, H., Daly P.W., \emph{A Guide to LaTeX},
Addison-Wesley, Reading, MA, 1999.
\bibitem{Pan} Pan, D., ``A Tutorial on MPEG/Audio Compression," \emph{IEEE
Multimedia}, Vol.2, pp.60-74, Summer 1998.
\end{thebibliography}
\end{document}