\documentclass[12pt]{article}
\author{\large ADITYA INGLE\\MIS : 112103050}
\date{\today}
\begin{document}
\title{\huge \bf Assignment 1}
\maketitle
\tableofcontents
\raggedright
\pagenumbering{gobble}
\newpage
\pagenumbering{roman}
\section{\textbf{Preliminaries}}
\vspace{5mm}
\subsection{\large Reason for studying concepts of programming language}
\paragraph{} It is natural for students to wonder how they will benefit from the study of programming language concepts. After all, many other topics in computer science 
are worthy of serious study. The following is what we believe to be a compelling 
list of potential benefits of studying concepts of \textbf {programming languages}.
\paragraph{} Increased capacity to express ideas. It is widely believed that the depth at which 
people can think is influenced by the expressive power of the language in 
which they communicate their thoughts.Those with only a weak understanding of natural language are limited in the complexity of their thoughts, 
particularly in depth of abstraction. In other words, it is difficult for people 
to conceptualize structures they cannot describe, verbally or in writing.
\subsubsection{Improved background for choosing appropriate languages.}
\subparagraph{} Some professional 
programmers have had little formal education in computer science; rather, 
they have developed their \textbf {\textit {programming skills}} independently or through 
in-house training programs. Such training programs often limit instruction 
to one or two languages that are directly relevant to the current projects 
of the organization. Other programmers received their formal training 
years ago. The languages they learned then are no longer used, and many 
features now available in programming languages were not widely known 
at the time.
\subsubsection{\textit{Increased ability to learn new languages.}}
\subparagraph{}Computer programming is still a relatively young discipline, and design methodologies and programming languages are still in a state of continuous evolution. This makes \textbf {software development} an exciting profession, but it also 
means that continuous learning is essential
\section{Programming Domains}
\paragraph{} Computers have been applied to a myriad of different areas, from controlling 
nuclear power plants to providing video games in mobile phones. Because of 
this great diversity in computer use, programming languages with very different 
goals have been developed.
\subsection{ Scientific Applications}
\subparagraph{} The first digital computers, which appeared in the late 1940s and early 1950s, 
were invented and used for scientific applications. Typically, the scientific applications of that time used relatively simple data structures, but required large 
numbers of floating- point arithmetic computations. The most common data 
structures were arrays and matrices; the most common control structures were 
counting loops and selections. The early high- level programming languages 
invented for scientific applications were designed to provide for those needs. 
Their competition was assembly language, so efficiency was a primary concern. 
The first language for scientific applications was Fortran. 
\subsubsection{ Business Applications}
The use of computers for business applications began in the 1950s. Special 
computers were developed for this purpose, along with special languages. The 
first successful high-level language for business was COBOL the initial version of which appeared in 1960.
\newpage
\pagenumbering{roman}
\end{document}

